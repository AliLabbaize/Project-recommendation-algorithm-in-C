\documentclass[11pt]{meetingmins}
\usepackage[utf8]{inputenc}   
\usepackage[T1]{fontenc}
\usepackage{tabularx}
\newcolumntype{C}{>{\centering}X}
\usepackage{colortbl}

\setcommittee{Projet C/SD 2018 : Compte-rendu de réunion 2}


\setmembers{
  Rémy BANEL,
  Ali LABBAIZE ,
  Paul-Louis FEULVARCH
}

\setdate{Mars 30, 2018}

\setpresent{
  Rémy BANEL,
  Ali LABBAIZE ,
  Paul-Louis FEULVARCH
}
\absent{ Aucun }
\begin{document}
\maketitle
\section{Type de réunion : \textnormal{Réunion de chantier} }

\section{Durée : \textnormal{1 heure 30 minutes} }
\section{Lieu : \textnormal{Salle de travail de l'école} }


\section{}
\section{Ordre du jour :}
\begin{hiddenitems}
\item
Présentation du début de l'état de l'art par A.~LABBAIZE
\item
Présentation de la bibliothèque graphique SDL par R.~BANEL
\item 
Mise au point sur les structures de données à utiliser 
\end{hiddenitems}

\section{Informations échangées :}
\begin{hiddenitems}
\item
\textbf{A.~LABBAIZE : } a présenté 4 systèmes de recommandation différents (cf. état d'art) se basant plus ou moins sur l'utilisateur et la communauté ainsi que le produit recommandé. Une approche hybride permettant d'allier les avantages de chaque système serait une option envisageable afin d'obtenir le meilleur système possible. \\
Ensuite il a présenté au groupe une proposition concernant les structures de données à utiliser (cf choix de conception dans le rapport), et le calcul matriciel à effectuer sur ces derniers. Afin d'optimaliser les temps de recherche, et proposer des résultats plus précis.
\item
La mise en évidence des risques temporels a également été abordée, ce qui a aboutit à la décision de faire un autre diagramme de Gantt en fonction de l'avancement du projet, afin de le comparer à la version prévisionnelle, dans l'optique d'une analyse post-mortem.
\item
\textbf{R. ~BANEL} propose d'utiliser la bibliotèque SDL afin de réaliser l'interface graphique. Tous les membres du projet devront donc télécharger celle-ci afin de pouvoir l'utiliser. Il sera également nécessaire de se former sur son utilisation.
\item
Brainstorming pour remplir une matrice SWOT (sur un document Google Docs)

\item Revue de l'atteinte des objectifs de la Todo list de la réunion précédente :
\begin{table}[h]
    \centering
    \begin{tabular}{|p{4cm}|c|c|c|c|}
    \hline
        \rowcolor{yellow} Description & Responsable & Validation
        \tabularnewline \hline
        Écrire l'état de l'art &  A.LABBAIZE & Oui.
        \tabularnewline \hline
        GANTT & PL.FEULVARCH & Non.
        \tabularnewline \hline
        Choix de l'interface graphique  & R.~Banel & Oui. \tabularnewline \hline 

        
    \end{tabular}
    \caption{Atteinte des objectifs}
    \label{tab:my_label}
\end{table}

\end{hiddenitems}

\newpage

\section{Décisions :}
\begin{itemize}
    \item Mettre en forme la matrice SWOT pour la prochaine réunion.
    \item Valider les structures de données mentionnées précédemment.
    \item Mettre en forme le compte-rendu de cette réunion sur ShareLaTeX afin de ne pas prendre de retard.
    \item Refaire un diagramme de Gantt à un stade plus avancé du projet.
\end{itemize}
\section{Todo list :}
\begin{table}[h]
    \centering
    \begin{tabular}{|p{4cm}|p{3cm}|c|c|c|}
    \hline
        \rowcolor{yellow} Description  & Responsable & Délai & Livrable & Validé par 
        \tabularnewline \hline
        
        Mise en forme de la matrice SWOT & R.~BANEL & Pour le 10/04/2018 & Une matrice SWOT & Toute l'équipe \tabularnewline \hline
        
        Faire un diagramme GANTT & PLF & Pour le 10/04/2018 & diagramme GANTT & Toute l'équipe
        \tabularnewline \hline 
        
        Coder des fonctions d'extraction de données à partir du ficher .txt donné sur Arche  & A.~LABBAIZE & Pour le 10/04/2018 & Code C & Toute l'équipe \tabularnewline \hline
        
        
        
    \end{tabular}
    \caption{Distribution des tâches}
    \label{tab:my_label}
\end{table}

%\section{Questions/Remarques : }
%\begin{itemize}
%    \item 
%\end{itemize}
\section{Date de la prochaine réunion : \textnormal{le 10 Avril 2018} }

\end{document}