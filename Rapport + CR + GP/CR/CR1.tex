\documentclass[11pt]{meetingmins}
\usepackage[utf8]{inputenc}   
\usepackage[T1]{fontenc}
\usepackage{tabularx}
\newcolumntype{C}{>{\centering}X}
\usepackage{colortbl}

%% Optionally, the following text could be set in the file
%% department.min in this folder, then add the option 'department'
%% in the \documentclass line of this .tex file:
%%\setcommittee{Department of Instruction}
%%
%%\setmembers{
%%  \chair{B.~Smart},
%%  B.~Brave,
%%  D.~Claire,
%%  B.~Gone
%%}

\setcommittee{Projet C/SD 2018 : Compte-rendu de réunion 1}


\setmembers{
  Rémy BANEL,
  Ali LABBAIZE ,
  Paul-Louis FEULVARCH
}

\setdate{Mars 15, 2018}

\setpresent{
  Rémy BANEL,
  Ali LABBAIZE
  Paul-Louis FEULVARCH
}
\absent{ Aucun }
\begin{document}
\maketitle
\section{Type de réunion : \textnormal{Réunion de lancement de projet et première répartition des tâches} }

\section{Durée : \textnormal{1 heure et 10 minutes} }
\section{Lieu : \textnormal{Médiathèque de l'école} }


\section{}
\section{Ordre du jour :}
\begin{hiddenitems}
\item
Partage des points de vue et des connaissances récoltées sur le sujet.
\item
Répartition des tâches en fonction des compétences de chacun.
\item
Se mettre d'accord sur l'environnement de travail (Code block, éditeur de texte ..etc)
\item
La mise en évidence des risques temporels. Définir les étapes du projet, afin de mettre en place un GANTT.
\item
Présentation du débbugeur GDB par A.~LABBAIZE au groupe.
\item
Election des responsables de projet.
\end{hiddenitems}

\section{Informations échangées :}
\begin{hiddenitems}
\item
Présentation de deux approches possibles des systèmes de recommandation, le "control-based learning" et le "collaborative filring" (cf état de l'art).
\item
Échange des informations récoltées sur le sujet:
\begin{itemize}
    \item \textbf{A.~LABBAIZE : }propose de mettre en place un système de notation qui se base sur 2 approches, une notation par critère (genre du film, acteur, réalisateur,...) et une notation basée sur le choix des utilisateurs. L'idée serait de créer un système se basant sur ces 2 notations.
    
    

    \item \textbf{R. BANEL : }parle de son expérience sur les projets en tant que R1. Trois grands conseils en sont ressortis :
    \begin{itemize}
        \item La mise en œuvre des outils de gestion de projet est primordiale
        \item Les tests sont obligatoires
        \item Il faut tout expliciter dans le rapport final, même ce qu'on va remplacer ou effacer
    \end{itemize}
    
    
    \item \textbf{PL. FEULVARCH : }explique que pour gagner en complexité, il serait possible d'utiliser des systèmes de recommandation prédéfinis.
    
\end{itemize}


\item Travail pour la semaine prochaine : 
\begin{itemize}
    \item Définir les lots de travail 
    \item Faire un diagramme de Gantt, puis un diagramme de PERT
    \item Commencer à écrire l'état d'art
    \item S'informer sur les interfaces graphiques en C (SDL,...) et les structures de données à utiliser
\end{itemize}


\end{hiddenitems}
\section{Décisions : }
\begin{itemize}
    \item A. LABBAIZE a été élu à l'unanimité chef de projet, R. BANEL sera quant à lui secrétaire
    \item Construire un diagramme GANTT/PERT
    \item Faire une réunion de chantier chaque semaine afin d'éviter l'effet tunnel
    \end{itemize}
\section{Todo list :}
\begin{table}[h]
    \centering
    \begin{tabular}{|p{4cm}|c|c|c|c|}
    \hline
        \rowcolor{yellow} Description & Responsable & Délai & Livrable & Validé par
        \tabularnewline \hline
        Etat d'art & A.LABBAIZE. & Pour le 30/03/2018 & Fichier LaTex & Toute l'équipe
        \tabularnewline \hline
        Se renseigner sur les interfaces graphiques & R. BANEL & Pour le 30/03/2018 & - & Toute l'équipe
        \tabularnewline \hline
        Réalisation du GANTT  & AL/PLF & Pour le 30/03/2018 & Diagramme & Toute l'équipe \tabularnewline \hline 
        
        
    \end{tabular}
    \caption{Distribution des tâches}
    \label{tab:my_label}
\end{table}

\section{Questions/Remarques :}
\begin{itemize}
    \item Afin de stocker des ressources à trier ou à stocker entre les réunions de projet, un drive d'équipe a été mis en place.
\end{itemize}
\section{Date de la prochaine réunion : \textnormal{le 30 Mars 2018} }



\end{document}