\documentclass[11pt]{meetingmins}
\usepackage[utf8]{inputenc}   
\usepackage[T1]{fontenc}
\usepackage{tabularx}
\newcolumntype{C}{>{\centering}X}
\usepackage{colortbl}

\setcommittee{Projet C/SD 2018 : Compte-rendu de réunion 5}


\setmembers{
  Rémy BANEL,
  Ali LABBAIZE ,
  Paul-Louis FEULVARCH
}

\setdate{Mai 15, 2018}

\setpresent{
  Rémy BANEL,
  Ali LABBAIZE
}
\absent{ Paul-Louis FEULVARCH }
\begin{document}
\maketitle
\section{Type de réunion : \textnormal{Réunion d'avancement} }

\section{Durée : \textnormal{30 minutes} }
\section{Lieu : \textnormal{Médiathèque de l'école} }


\section{}
\section{Ordre du jour :}
\begin{hiddenitems}

Mettre en commun le travail après les vacances.
\end{hiddenitems}



\section{Informations échangées :}
\begin{hiddenitems}
\item
\textbf{A.~LABBAIZE : } présente la finalisation du code d'extraction. Toutes les informations importantes sont disponibles. Des | ont dû être ajouté afin d'obtenir les genres (aventures, comédie,...) des films.

\item
\textbf{R. ~BANEL} parle des structures à définir. Pose la question de la pertinence de créer une structure matrice. 


\item Revue de l'atteinte des objectifs de la Todo list de la réunion précédente :
\begin{table}[h]
    \centering
    \begin{tabular}{|p{4cm}|c|c|c|c|}
    \hline
        \rowcolor{yellow} Description & Responsable & Validation
        \tabularnewline \hline
        Fonction d'extraction du fichier .txt &  A.LABBAIZE & Oui.
        \tabularnewline \hline
        Avancement de l'interface graphique & PL.FEULVARCH & Oui.
        \tabularnewline \hline
        Produit matriciel  & R.~Banel & Oui. \tabularnewline \hline 

        
    \end{tabular}
    \caption{Atteinte des objectifs}
    \label{tab:my_label}
\end{table}

\end{hiddenitems}

\newpage

\section{Décisions :}
\begin{itemize}
    \item Afin d'avancer plus rapidement dans le projet, il a été décidé d'effectuer de réunion de travail. Ceci permettant de travailler à 3 et de pouvoir s'appuyer sur les autres membres du groupe en direct en cas de blocage ou de question.
\end{itemize}
\section{Todo list :}
\begin{table}[h]
    \centering
    \begin{tabular}{|p{4cm}|p{3cm}|c|c|c|}
    \hline
        \rowcolor{yellow} Description  & Responsable & Délai & Livrable & Validé par 
        \tabularnewline \hline
        
        Récupérer les affiches des 99 films et séries de la liste & R.~BANEL & Pour le 17/05/2018 & image & Toute l'équipe \tabularnewline \hline
        
        
        Réfléchir aux différentes structures utiles & A.~LABBAIZE & Pour le 17/05/2018 & - & Toute l'équipe \tabularnewline \hline
        
        
        
    \end{tabular}
    \caption{Distribution des tâches}
    \label{tab:my_label}
\end{table}

%\section{Questions/Remarques : }
%\begin{itemize}
%    \item 
%\end{itemize}
\section{Date de la prochaine réunion : \textnormal{le 17 Mai 2018} }

\end{document}