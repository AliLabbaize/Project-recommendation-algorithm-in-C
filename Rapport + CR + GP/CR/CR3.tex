\documentclass[11pt]{meetingmins}
\usepackage[utf8]{inputenc}   
\usepackage[T1]{fontenc}
\usepackage{tabularx}
\newcolumntype{C}{>{\centering}X}
\usepackage{colortbl}

\setcommittee{Projet C/SD 2018 : Compte-rendu de réunion 3}


\setmembers{
  Rémy BANEL,
  Ali LABBAIZE ,
  Paul-Louis FEULVARCH
}

\setdate{Avril 10, 2018}

\setpresent{
  Rémy BANEL,
  Ali LABBAIZE ,
  Paul-Louis FEULVARCH
}
\absent{ Aucun }
\begin{document}
\maketitle
\section{Type de réunion : \textnormal{Réunion de chantier} }

\section{Durée : \textnormal{1 heure 00 minutes} }
\section{Lieu : \textnormal{Salle de travail de l'école} }


\section{}
\section{Ordre du jour :}
\begin{hiddenitems}
\item
Présentation du code d'extraction des données contenues dans le fichier .txt par A.~LABBAIZE
\item
Présentation du GANTT par PL.~FEULVARCH
\item 
Mise au point sur les structures de données à garder et à utiliser dans l'extraction du fichier .txt 
\end{hiddenitems}

% à completer
%%%%%%%%%%%%%%%%%%%%%%%%%%

\section{Informations échangées :}
\begin{hiddenitems}
\item
\textbf{A.~LABBAIZE : } a présenté ses idées afin d'extraire les données du fichier texte. Il met en évidence les difficultés qu'il a rencontrées afin de pouvoir séparer le titre du film, son genre, son réalisateur et toute autre information utile.
\item
\textbf{PL.~FEULVARCH : }a présenté le diagramme GANTT qu'il a réalisé. Les lots de travail semblent bien définis, cependant la mise en page de ce GANTT pose problème afin de pouvoir l'intégrer dans le rapport final. Il sera nécessaire de le mettre en forme afin de le garder clair et lisible dans la version finale du rapport.
\item
\textbf{R. ~BANEL} a présenté la matrice SWOT.


\item Revue de l'atteinte des objectifs de la Todo list de la réunion précédente :
\begin{table}[h]
    \centering
    \begin{tabular}{|p{4cm}|c|c|c|c|}
    \hline
        \rowcolor{yellow} Description & Responsable & Validation
        \tabularnewline \hline
        Fonction d'extraction du fichier .txt &  A.LABBAIZE & Partiellement.
        \tabularnewline \hline
        GANTT & PL.FEULVARCH & Oui.
        \tabularnewline \hline
        SWOT  & R.~Banel & Oui. \tabularnewline \hline 

        
    \end{tabular}
    \caption{Atteinte des objectifs}
    \label{tab:my_label}
\end{table}

\end{hiddenitems}

\newpage

\section{Décisions :}
\begin{itemize}
    \item Il est important de décider des délimiteurs dans le fichier texte afin de pouvoir obtenir les informations souhaitées. Il a par exemple été décidé de prendre tout ce qui est compris entre le symbole "." et le symbole "(" comme étant des titres de films. 
    \item Refaire le GANTT sous une autre forme afin de le rendre plus lisible.
    \item Terminer l'état de l'art.
    \item Commencer à se familiariser avec les interfaces graphiques en essayant de coder quelques fonctions simples afin d'afficher des images et des boutons cliquables.
\end{itemize}
\section{Todo list :}
\begin{table}[h]
    \centering
    \begin{tabular}{|p{4cm}|p{3cm}|c|c|c|}
    \hline
        \rowcolor{yellow} Description  & Responsable & Délai & Livrable & Validé par 
        \tabularnewline \hline
        
        Continuer l'état de l'art & R.~BANEL & Pour le 20/04/2018 & Fichier LateX & Toute l'équipe \tabularnewline \hline
        
        Coder une première interface graphique avec SDL & PLF & Pour le 20/04/2018 & Code C & Toute l'équipe
        \tabularnewline \hline 
        
        Coder des fonctions d'extraction de données à partir du ficher .txt donné sur Arche  & A.~LABBAIZE & Pour le 20/04/2018 & Code C & Toute l'équipe \tabularnewline \hline
        
        
        
    \end{tabular}
    \caption{Distribution des tâches}
    \label{tab:my_label}
\end{table}

%\section{Questions/Remarques : }
%\begin{itemize}
%    \item 
%\end{itemize}
\section{Date de la prochaine réunion : \textnormal{le 20 Avril 2018} }

\end{document}